\documentclass{article}
\usepackage[utf8]{inputenc}

\title{Notes on Principles of Layered Attestation by Paul Rowe}
\author{arfritz97 }
\date{August 2019}

\begin{document}

\textbf {Limitations of Measurement}

Measurement cannot prevent corruption; at best it can only detect corruption.

\textbf {Strategy for measurement}

The trust system should be based on a bottom up chain of measurement. The "recent or deep" theorem validates this. The theorem states "If a system has measured deeper components before more shallow ones, then the only way for an adversary to corrupt a component t without detection is either by recently corrupting one of t's dependencies, or else by corrupting a component even deeper in the system. 

\textbf {Bundling evidence}

The order of measurement is important so the result of attestation must convey order. 

Problem: Structure of TPM quote does not always reflect ordering information, especially if some of the components depositing measurement values might be dynamically corrupted. 

Thus, the following bundling theorem, Theorem 3, states: "If the system satisfies certain assumptions, and the TPM quote formed according to our bundling strategy indicates no corruptions, then either the measurements were really taken bottom-up, or the adversary recently corrupted one of t's dependencies, or else the adversary corrupted an even deeper component. 

\section {Example}

Consider an enterprise that would like to ensure that systems connecting to its network provide a fresh system scan by the most up-to-date virus checker. Suppose all systems have A1, a component capable of accurately reporting the running version of the virus checker, and A2, a component capable of measuring the runtime state of the kernel. Assume A1 and A2 are both measured by roots of trust (rtm) as part of secure boot. The system then contains {sys, vc, ker, A1, A2, rtm} where sys represents the parts of the system, vc represents the virus checker, and ker represents the kernel.

\section {Questions from paper / takeaways}

\begin{itemize}
\item Rowe modeled the set of events as a poset. A poset is just a parially ordered set yet he said if there is an element then it has a unique maximal element. What do we think about a poset?
\item The whole measurement system is defined as a tuple. Should we define the negotiation system as a tuple?
  \begin{itemize}
  \item MS = (O, M, C)
  \item M = the measures relation (represents who can measure who)
  \item C = the context relation
  \item O = the set of objects (software components)
  \end{itemize}
\item Thye bounded events in time with att-start(n). Do we need to think about bounding events in time? 
\end{itemize}


\end{document}
