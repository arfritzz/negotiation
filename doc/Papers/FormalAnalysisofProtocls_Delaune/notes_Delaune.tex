
\documentclass{article}
\usepackage[utf8]{inputenc}

\title{Notes on Formal Analysis of Protocols based on TPM State Registers}
\author{arfritz97 }
\date{August 2019}

\begin{document}

\section {Rules used to model the Alice/Bob situation}

Alice and Bob are sharing secrets. Alice has two secrets and
Bob can choose one or the other, but not both. Alice cannot 
alter the secrets based on Bob's decision.


\textbf {Attacker Rules}

\begin{itemize}
 
One must first assume protocols are executed in the presence
of an attacker. This attacker can intercept all messages,
compute new messages from the messages it has recieved,
and send any message it can build.

\end{itemize}

\textbf {TPM Rules}

\begin{itemize}

\item CertifyKey - command that allows one to obtain a certificate
on a key that is stored in teh device. 

\item Unbind -- command allows one to retieve the contents of an
encryption provided that the decryption key is stored in 
the key table of the TPM. This can only be executed if the 
PCR's current value is the one specified in the key table.

\item Extend -- If the attacker is able to extend teh PCR with value
in the state and the attacker knows some term in the state then
the attacker still knows that term after having extended
the PCR. Also, the key table is maintained after when the 
PCR is extended.

\end{itemize}

\textbf {Protocol Rules}
\begin{itemize}
If the attacker can provide a certificate of a key bound to 
the PCR value then Alice will encrypt the first secret using 
this key. The second rule is similar for the second secret.
\end{itemize}

\end{document}
