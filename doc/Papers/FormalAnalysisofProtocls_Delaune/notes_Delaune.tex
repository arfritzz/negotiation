
\documentclass{article}
\usepackage[utf8]{inputenc}

\title{Notes on Formal Analysis of Protocols based on TPM State Registers}
\author{arfritz97 }
\date{August 2019}

\begin{document}

\section {Rules used to model the Alice/Bob situation}

\textbf {Attacker Rules}

One must first assume protocols are executed in the presence
of an attacker. This attacker can intercept all messages,
compute new messages from the messages it has recieved,
and send any message it can build.

\textbf {TPM Rules}

CertifyKey -> command that allows one to obtain a certificate
on a key that is stored in teh device. 

Unbind -> command allows one to retieve the contents of an
encryption provided that the decryption key is stored in 
the key table of the TPM. This can only be executed if the 
PCR's current value is the one specified in the key table.

Extend -> If the attacker is able to extend teh PCR with value
in the state and the attacker knows some term in the state then
the attacker still knows that term after having extended
the PCR. Also, the key table is maintained after when the 
PCR is extended.

\textbf {Protocol Rules}

If the attacker can provide a certificate of a key bound to 
the PCR value then Alice will encrypt the first secret using 
this key. The second rule is similar for the second secret.

\end{document}
