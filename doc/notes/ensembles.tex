\title{Ensembles}
\author{
        Anna Fritz
}
\date{\today}

\documentclass[12pt]{article}

\begin{document}
\maketitle

\begin{abstract}
Notes about ensembles. \ldots
\end{abstract}

\section{Information}

Formally, (per wiki) an ensemble is an idealization consisting of a large number of virtual copies of a system considered all at once, each of which represents a possible state that the real system might have been in.

A use of \boldsf{ensembles} is Coq is as a data type. With that, ensemble is synonymous with set. So if we have a type U:Set we can say U is of type ensemble. Then, if we have other variables, A and B that are of type Ensemble U, we are saying that A and B are subsets of U.   

An example of this would be to say `Included U (Intersection U A B) (Intersection U B A)' to say that the intersection of the subsets of A and B are in the overarching set, U.

\section{Defintions}

\bibliographystyle{abbrv}
\bibliography{main}

\end{document}


This is never printed
~\cite{Gil:02}.

\paragraph{Outline}
The remainder of this article is organized as follows.
Section~\ref{previous work} gives account of previous work.
Our new and exciting results are described in Section~\ref{results}.
Finally, Section~\ref{conclusions} gives the conclusions.
