\documentclass[10pt]{report}
\usepackage{tikz}
\usetikzlibrary{arrows}
\usepackage{natbib}
\usepackage{graphicx}
\usepackage{url}
\usepackage{fancyhdr}
\pagestyle{fancy}

\lhead{CAPTools Documentation}
\rhead{KU System-Level Design Group}
\lfoot{\copyright The University of Kansas, 2019}
\cfoot{\thepage}

\newtheorem{conjecture}{Conjecture}
\newtheorem{obligation}{Obligation}
\newtheorem{definition}{Definition}

\usepackage[textsize=tiny]{todonotes}
\usepackage{ifthen}
% \newboolean{submission}  %%set to true for the submission version

\newcommand{\squash}{\itemsep=0pt\parskip=0pt}

\parskip=\medskipamount
\parindent=0pt

\bibliographystyle{abbrvnat}

\begin{document}

\title{Certified Attestation Protocol Tools - Version 0.1}
\author{Anna Fritz \\
  Information and Telecommunication Technology Center \\
  The University of Kansas \\
  \url{arfritzz@ku.edu}
}

\section {Inital Overview}

  Negotiation occurs between two parties: the appraiser and the target. 
  The appraiser sends the target a \textbf{request}. The target then responds
  with a \textbf{proposal} which is a set of \textbf{protocols}.

\section {Request}

\subsection {Certificate Authority}
  
  As part of ISAKMP, a certificate authority is needed for strong 
  authentication of a communicating peer (ie the appraiser and the
  target). 

\subsection {Request Composition}
  
  The request will be composed of ________. 

\section {Proposal}

\subsection {Producing a proposal}

  The function, ($\gamma_{n}$), selects a protocol (P_{k}) from a proposal
  $\langle p_0,p_1,\ldots,p_n\rangle$. In other words, there must be some
  way to take the request and generate a protocol. 
   
\section {Selecting Proposal}

\section {Evaluating Propsal}

\section {Questions:}
\begin{enumerate}
  \item What does the certificate authority get us? A secure channel but 
        does it say anything about the appraiser or target's
        privacy policy?
  \item How does the request generate a proposal? 
  \begin{itemize}
    \item Could think of a request as a condition like all even numbers.
    \item Then Proposal consists of many different sets composed of even
          and odd numbers with each set having varying amounts of numbers.
    \item Then what filters the set to include only even numbers?
          Is that $\delta_c$?
    \item $\delta_c$ is a functor that transforms proposals into evidence.
          I dont really understand this at all.  
  \end{itemize}
  \item Is the targets response, the proposal, an ordered list?
        I think we need a function to ensure ordering.
\end{document}

\maketitle

\todototoc\listoftodos
\listoffigures
\listoftables
