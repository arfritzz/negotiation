\documentclass[10pt]{article}
\usepackage{tikz}
\usetikzlibrary{arrows}
\usepackage{natbib}
\usepackage{graphicx}
\usepackage{url}
\usepackage{fancyhdr}
\pagestyle{fancy}

\lhead{CAPTools Documentation}
\rhead{KU System-Level Design Group}
\lfoot{\copyright The University of Kansas, 2019}
\cfoot{\thepage}

\newtheorem{conjecture}{Conjecture}
\newtheorem{obligation}{Obligation}
\newtheorem{definition}{Definition}

\usepackage[textsize=tiny]{todonotes}
\usepackage{ifthen}
% \newboolean{submission}  %%set to true for the submission version

\newcommand{\squash}{\itemsep=0pt\parskip=0pt}

\parskip=\medskipamount
\parindent=0pt

\bibliographystyle{abbrvnat}


\title{Certified Attestation Protocol Tools - Version 0.1}
\author{Anna Fritz \\
  Information and Telecommunication Technology Center \\
  The University of Kansas \\
  \url{arfritzz@ku.edu}
}

\begin{document}

\section {Initial Overview}

  Negotiation occurs between two parties: the appraiser and the target. 
  The appraiser sends the target a \textbf{request}. The target then responds
  with a \textbf{proposal} which is a set of \textbf{protocols}.

\section {Request}

\subsection {Certificate Authority}
  
  As part of ISAKMP, a certificate authority is needed for strong 
  authentication of a communicating peer (ie the appraiser and the
  target). 

\subsection {Request Composition}
  
  Right now the exact understanding of the Request is not important.
  Only a general understanding in needed and useful. 
  
  A request is composed of some sort of evidence. It can be three things, either evidence, a sum of evidence, or a product of evidence. However, sum and product are the mathematical definitions were sum equates to either (the boolean or) and product equates to both (the boolean and). 
  
  At this point, there really is not much else that is needed for a Request; all that is needed is evidence. In the future, some things that should be considered in the request is:
  
  \begin{itemize}
   \squash
   \item place
   \item the appraiser privacy policy
   \item ISAKMP
  \end{itemize}

\section {Proposal}

\subsection {Producing a proposal}

  A proposal is a set of protocol generated by the target upon receiving the target's request. Therefore, the proposal takes in the appraiser's request and returns a list of terms. The coq definition for this may look something like an is considered an interpreter: 
  
  \begin{verbatim}
  Definition propose : (request ev) -> list term
  \end{verbatim}  
  
  In this definition, the appraiser receives a list of terms. The list of terms must satisfy the privacy policy for the target before being sent to the appraiser. 

\section {Selecting Proposal}

  The function, ($\gamma_{n}$), selects a protocol (P_{k}) 
  from a proposal $\langle p_0,p_1,\ldots,p_n\rangle$.  

  The selection of a Proposal will involve ensuring that the chosen protocol meets the initial negotiation condition. This can be represented in an Inductive definition as follows:
  
  \begin{verbatim}
  Inductive negotiationR : Request -> Place -> term -> Prop :=
  | n1 : (ev1) -> n -> (USM 1) -> True
  .
  .
  .
  . where one can list various options for negotiation. 
  \end{verbatim} 
  
  
  
\section {Evaluating Proposal}

\section {Questions:}
\begin{enumerate}
  \item What does the certificate authority get us? A secure channel but 
        does it say anything about the appraiser or target's
        privacy policy?
  \item How does the request generate a proposal? 
  \begin{itemize}
    \item Could think of a request as a condition like all even numbers.
    \item Then Proposal consists of many different sets composed of even
          and odd numbers with each set having varying amounts of numbers.
    \item Then what filters the set to include only even numbers?
          Is that $\delta_c$?
    \item $\delta_c$ is a functor that transforms proposals into evidence.
          I don't really understand this at all.  
  \end{itemize}
  \item Is the targets response, the proposal, an ordered list?
        I think we need a function to ensure ordering.
\end{enumerate}

\section {Examples}

Throughout the process of understanding negotiation, there are many examples that have helped me get a better grasp on Coq and what Negotiation entails. 

\subsection {The Fruit Example: understanding constructing values}

Let Fruit be a set such that Fruit = { apple , orange , pear }. Then an inductive data structure for Fruit could read:  

\begin{verbatim}

Inductive request : Type := 
 
 | one n -> fruit

 | prod fruit -> fruit -> fruit

 | sum fruit -> fruit -> fruit.
\end{verbatim}

Where prod is equivalent to the boolean condition AND and sum is equivalent to the boolean condition OR. Then creating examples of this would look like: 


(one apple)

(prod (one apple) (one pear))

(sum (one apple) (one pear))

(prod ((prod (one apple) (one pear)) one apple)


Therefore, one apple is a constructing value. It creates a new element that is now part of the data structure. Overall, this Inductive definition of Request is a "little language."

Then, generalizing this to all data type we consider the problem of wanting to use the request data struct for a McDonald's order. If the structure was untyped, then one could just request an order. To do this, implement the following structure. 

\begin{verbatim}
Inductive request (ev : Type) : Type :=

| one n -> ev

| prod ev -> ev -> ev

| sum ev -> ev -> ev

\end{verbatim}

\end{document}

